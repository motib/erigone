\documentclass[11pt]{article}
\usepackage{mathpazo}
\usepackage{url}
\usepackage[pdftex]{graphicx}
\usepackage{verbatim}

\newcommand{\vn}{\textsc{VN}}
\newcommand{\jf}{\textsc{jflap}}
\newcommand{\dt}{\textsc{dot}}
\newcommand{\spn}{\textsc{Spin}}
\newcommand{\js}{\textsc{jSpin}}
\newcommand{\prm}{\textsc{Promela}}
\newcommand{\p}[1]{\texttt{#1}}
\newcommand{\bu}[1]{\textsf{#1}}

\textwidth=15cm
\textheight=20cm
\topmargin=0pt
\headheight=0pt
\oddsidemargin=1cm
\headsep=0pt
\renewcommand{\baselinestretch}{1.1}
\setlength{\parskip}{0.20\baselineskip plus 1pt minus 1pt}
\parindent=0pt

\title{\vn{} - Visualization of Nondeterminism\\User's Guide\\\mbox{}\\\large{Version 2.3.1}}
\author{Mordechai (Moti) Ben-Ari\\
Department of Science Teaching\\
Weizmann Institute of Science\\
Rehovot 76100 Israel\\
\textsf{http://stwww.weizmann.ac.il/g-cs/benari/}}
%\date{}
\begin{document}

\maketitle
\thispagestyle{empty}

\vfil

\begin{center}
Copyright (c) 2006-8 by Mordechai (Moti) Ben-Ari.
\end{center}
Permission is granted to copy, distribute and/or modify this document
under the terms of the GNU Free Documentation License, Version 1.2
or any later version published by the Free Software Foundation;
with Invariant Section ``Introduction,'' no Front-Cover Texts, and no Back-Cover Texts.
A copy of the license is included in the file \p{fdl.txt}
included in this archive.
\vfil

\newpage

\section{Introduction}

\vn{} is a tool for studying the behavior of nondeterministic finite automata 
(NDFA). It takes a description of an automaton and generates a nondeterministic 
program; the program can then be executed randomly or guided interactively. The 
automaton and the execution path are graphically displayed.

\vn{} is written in Java for portability. It is based on other software tools: 
\begin{itemize}
  \item The interactive editor of \jf{} is used to 
create an automaton, which is saved as an XML file that is the input to \vn{}. See: 
Susan H. Rodger and Thomas W. Finley. \textit{JFLAP: An Interactive Formal 
Languages and Automata Package}. Jones \& Bartlett, 2006. The website is 
\url{http://jflap.org/}.
\item The program generated from the automaton is written in 
\prm{}, the language of the \spn{} model checker. See: Gerard J. Holzmann. 
\textit{The Spin Model Checker: Primer and Reference Manual}. Addison-Wesley, 
2004. The website is \url{http://spinroot.com}.
\item The graphical description of the automaton and path are created in the \dt{} 
language and layed out by the \dt{} tool. Graphs in PNG format are created and
are then displayed within \vn{}. \dt{} is part of the \textsc{Graphviz} 
package; see: \url{http://graphviz.org/}.
\item An ANSI C compiler for compiling a \spn{} verifier; this is used to find
an accepting computation in the automaton. For Windows you can use
\p{MinGW}  \url{http://mingw.org} or \p{Cygwin} \url{http://cygwin.com}.
\end{itemize}

The \vn{} software is copyrighted according the the GNU General Public License.
See the files \p{copyright.txt} and \p{gpl.txt} in the archive.

The \vn{} webpage is \url{http://stwww.weizmann.ac.il/g-cs/benari/vn/}.

The design of \vn{} is explained in:
M. Ben-Ari. \textit{Principles of the Spin Model Checker}, Springer, 2008.

\newpage

\section{Installation and execution}

\begin{itemize}
\item Install the Java SDK or JRE (\url{http://java.sun.com}).
\textbf{\vn{} needs Java 1.5 at least.}

\item Simplified installation for Windows:
\begin{itemize}
\item Download the \vn{} installation file called \p{vn-N.exe},
where \p{N} is the version number.
Execute the installation file.
\item Create a directory \verb=c:\mingw= and download the C 
compiler archive \p{mingw.exe} into that directory. Execute this self-extracting file.
\end{itemize}

\item Custom installation:
\begin{itemize}
\item Download the \vn{} distribution file called \p{vn-N.zip},
where \p{N} is the version number.
Extract the files into a clean directory.
\item Install \spn{}, \dt{} and \jf{}.
(\jf{} optional and is only needed if you want to invoke it from within \vn{}.)
To simplify installation, a minimal set of files 
for running these tools in Windows is included in the archive \p{vn-bin.zip}.
\item Install an ANSI C compiler.
\item Modify the configuration file \p{config.cfg} to reflect the
locations of the programs.
\end{itemize}

\item The installation will create the following subdirectories: \p{docs} for the
documentation; \p{vn} for the source files; \p{bin} for the libraries and executables
for \spn{}, \dt{} and \jf{}; \p{txts} for the text files
(help, about and copyright); and \p{examples} for example programs.

\item To run \vn{}, execute the command \p{javaw -jar vn.jar}.
An optional argument names the \p{jff} file from \jf{} to be opened initially.
A batch file \p{run.bat} is supplied which contains this command.

\item Configuration data for \vn{} is given in the source file \p{Config.java}, 
as well as in the file \p{config.cfg}. The latter is reset to its default values
if it is erased.

\item To rebuild \vn{}, execute \p{build.bat}, which will compile all the source
files and create the file \p{vn.jar} with the manifest file.
\end{itemize}

\newpage

\section{Interacting with \vn{}}

The display of the \vn{} program is divided into three scrollable panes. 
Messages from \vn{} are displayed in the text area at the bottom of the screen.
Graphs of automata are displayed in the left-hand pane and graphs of the paths 
are displayed in the right-hand pane. The size of the panes is adjustable and 
the small triangles on the dividers can be used to maximize a pane.
Above the panes is another text area called the path area
where execution paths of the automaton are displayed.

Interaction with VN is through a toolbar. All toolbar buttons have mnemonics 
(Alt-character).

\begin{figure}
\begin{center}
\includegraphics[width=.9\textwidth,keepaspectratio=true]{vn.png}
\end{center}
\end{figure}
 
\begin{description}
\item[\bu{Open}] Brings up a file selector; select a \p{jff} file and the 
automaton described in the file will be displayed.
\item[\bu{Edit}] This invokes \jf{} on the current automaton.
\textbf{The modified \p{jff} file must be re-opened in \vn{} after it is
saved in \jf{}.}
\item[\bu{String or length}] Enter an input string for the automaton in this text field.
See below on entering a length.
\item[\bu{Generate}] Once an automaton has be opened and an input string entered in 
the text field, selecting \bu{Generate} will create a program to execute the 
automaton.
\item[\bu{Random}] This will execute the program for the automaton with a 
random resolution of nondeterminism. The sequence of states will appear in the 
text area below the toolbar, along with an indication if the input string was 
accepted or rejected.
\item[\bu{Create}] This will execute the program for the automaton, resolving 
nondeterminism interactively. Determinisitic choices will be made 
automatically; if a choice cannot be taken, it will be displayed in square 
brackets and cannot be selected. \bu{Quit} terminates the execution. Keyboard 
shortcuts: \bu{Tab} moves between buttons and \bu{Space} or \bu{Enter} selects the 
highlighted button.
\item[\bu{Find}] This searches for an accepting computation of the automaton.
\item[\bu{Next}] This searches for the next accepting computation of the automaton.
When no more accepting computations exist, the number of accepting computations
is displayed in the path area.
\item[\bu{DFA}] See below.
\item[\bu{Options}] Displays a diaglog for choosing the size of the graphs
(small, medium, large) and whether the highlighting of the paths will be in
color or bold. When you select \bu{OK} the changes will be saved, 
along with the directory of the current open file.
(The options \bu{Interactive} and \bu{Choose} concern generating
interactive programs as discussed below.)
\item[\bu{Help}] Displays a short help file.
\item[\bu{About}] Displays copyright information about the software. The 
version number appears in the title bar.
\item[\bu{Exit}] Exit the application.
\end{description}
Once a computation has been found , the path itself is displayed
in the right-hand pane and highlighted
in the automaton in the left-hand pane. The difference is that the 
``path'' includes multiple visits to the same state, while the highlighted path 
will, of course, include just one state for all occurences.

\newpage 

\noindent\textbf{Searching for all inputs that are accepted}

If you an integer value in the text field \bu{String or length},
each selection of \bu{Next}
searches for an accepting computation for \emph{any} input string of
this length. When no more accepting computations are found,
the set of strings that are accepted is displayed in path area.
For the automaton \p{ndfa.jff} supplied in the archive and with input length six, 
there are seven accepting computations for the four inputs:
\p{aaaaaa, aaaabb, aaabbb, aabbbb}.

\bigskip
\bigskip

\noindent\textbf{Equivalence classes of deterministic finite automata}

If the automaton is  deterministic, it is possible to obtain
the partition of a set of input strings of a given length
into the equivalence classes 
associated with each state.\footnote{This functionality was inspired by \bu{ProofChecker}:\\ \hspace*{2cm}\url{http://research.csc.ncsu.edu/accessibility/ProofChecker/index.html}.}
Enter a length in the text field and select \bu{Generate}.
Now select \bu{DFA} repeatedly;
for each state, starting from the first state, the set of input strings ``accepted'' by that
state is displayed. When \bu{DFA} has been selected for all states,
the set of equivalence classes is displayed in the right-hand pane.\footnote{\bu{Generate} 
can be selected at any time to reset to the first state.}

For the DFA in the \p{examples} directory, the equivalence classes
of length~$4$ are:
\begin{verbatim}
q0: [bbaa, baba, abba, aaaa]
q1: [bbab, babb, abbb, aaab]
q2: [bbba, baaa, abaa, aaba]
q3: [bbbb, baab, abab, aabb]
\end{verbatim}

\bigskip
\bigskip

\noindent\textbf{Generating interactive programs}

VN can generate standalone interactive programs from a FA.
Select a language from \bu{Options/Interactive}: \prm{}, \textsc{Java},
or \textsc{Jeliot} for the Jeliot program animation system 
(\url{http://cs.joensuu.fi/jeliot/}). The Jeliot version is in Java,
but contains non-standard constructs that simplify the animation.

Nondeterminism makes sense only for \prm{} programs;
Java programs generated for NDFAs resolve nondeterminism in a fixed order.
Select \bu{Options/Choose NDFA} to generate a program whose input is
a \emph{transition number} rather than an input symbol.
As can be seen from the generated source code, 
the transitions are given in lexicographic order.

This feature can also be invoked from the command line:
\begin{verbatim}
java -jar vn.jar file-name [promela|java|jeliot] [choose]
\end{verbatim}

%\newpage

\section{Files}
The different phases of processing \vn{} communicate through files. Here is a
list of the extensions of these files:
\begin{itemize}
  \item[\p{jff}] JFLAP XML description of an automaton
  \item[\p{pml}] Promela source generated from the automaton
  \item[\p{pth}] Path resulting from running \spn{} on the \p{pml} file
  \item[\p{dot}] Graphics file describing automaton and path
  \item[\p{png}] \textsc{PNG} graphics file after layout by \dt{}
\end{itemize}

%\newpage

\section{Software structure}
\p{VN} is the main class and contains declarations
of the GUI components and the event handler for all the buttons.

\p{Config} contains constants and properties.

\p{Options} displays a dialog for changing the size and highlight of the graphs.

\p{DisplayFile} displays the files for \bu{About} and \bu{Help}
in a \p{JFrame}.

\p{JFFFileFilter} is used with a \p{JFileChooser} to open a \p{jff} file.

\p{ImagePanel} displays the \textsc{PNG} files.

\p{GenerateSpin} generates the \prm{} program from the automaton.
In the generated program, the variable \p{x} is used to ensure that all
states of the state space are explored, even if the verification
revisits a state of the automaton several times. The assignment to
\p{\_} is used to ensure that \p{x} is included in the state vector.

\p{ReadXML} reads the XML description of the automaton from the \p{jff} file and
stores the data in \p{states} and \p{transitions}. These types are defined in
classes \p{State} and \p{Transition}.

\p{ReadPath} reads the path data printed by the \prm{} program and stores it in
\p{pathStates} and \p{pathTransitions}.

\p{WriteGraph} writes the automaton and paths in \dt{} format and forks a process to
run \dt{} to layout the graph.

\p{RunSpin} forks a process to execute \spn{}. In interactive mode, it reads the
\prm{} program so that the \p{JOptionPane} used for selecting among
nondeterministic choices can display actual source code. For \bu{Find} and
\bu{Next} processes are forked to run the C compiler and \p{pan}. 
For \bu{Next}, \p{pan} is run with the \p{-c} argument.

%\newpage

\appendix

\section{Release notes}

\begin{description}
\item[2.3.1] Fixed bug with \bu{Next} by adding \p{goto halt} to accept
state.
\item[2.3] As of Version 6.3, \jf{} as distributed can be invoked from another
program so a modified version is no longer needed.
\item[2.2] Generation of interactive programs.
\item[2.1] Self-installing archive and relative path names.
\item[2.0] Invocation of \jf{}.
\item[1.8] \bu{DFA} function added. Improved error messages.
\item[1.7] In interactive mode (\bu{Create}) the incremental path is displayed
graphically as well as in the path area.
\item[1.6] The input length can be entered instead of a string;
\vn{} searches for all inputs of this length that are accepted.
The user interface has been slightly modified to support this feature.
\item[1.5] \bu{Next} is implemented. The paths are now shown automatically.
The \bu{Show} command can be restored by changing a constant
in \p{Config.java}.
\item[1.4] Configuration file added. Bug fix in display of path.
\item[1.3] Improvement in generated \prm{} program's treatment of \p{timeout}.
\item[1.2] Added \bu{Find}.
\item[1.1] Bug fixes and \bu{Options}.
\item[1.0] Initial public release.
\end{description}

\end{document}
