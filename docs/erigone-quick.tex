\documentclass[11pt]{article}
\usepackage{mathpazo}
\usepackage{url}
\usepackage{verbatim}

\newcommand*{\prg}{\textsc{Erigone}}
\newcommand*{\eui}{\textsc{EUI}}
\newcommand*{\spn}{\textsc{Spin}}
\newcommand*{\jsp}{\textsc{jSpin}}
\newcommand*{\prm}{\textsc{Promela}}
\newcommand*{\p}[1]{\texttt{#1}}
\newcommand*{\bu}[1]{\textsf{#1}}

\textwidth=15cm
\textheight=22.5cm
\topmargin=0pt
\headheight=0pt
\oddsidemargin=1cm
\headsep=0pt
\renewcommand{\baselinestretch}{1.1}
\setlength{\parskip}{0.20\baselineskip plus 1pt minus 1pt}
\parindent=0pt

\title{The \prg{} Model Checker\\\bigskip%
Quick Start Guide}
\author{Mordechai (Moti) Ben-Ari\\
Department of Science Teaching\\
Weizmann Institute of Science\\
Rehovot 76100 Israel\\
\textsf{http://stwww.weizmann.ac.il/g-cs/benari/}}
%\date{}
\begin{document}
\maketitle
\thispagestyle{empty}

\vfill

\begin{center}
Copyright \copyright{} 2010-12 by Mordechai (Moti) Ben-Ari.
\end{center}
This work is licensed under the Creative Commons Attribution-ShareAlike 3.0
License. To view a copy of this license, visit
\url{http://creativecommons.org/licenses/by-sa/3.0/}; or, (b) send a letter
to Creative Commons, 543 Howard Street, 5th Floor, San Francisco,
California, 94105, USA.

\newpage

\section{Introduction}

\prg{} is a \emph{model checker} that can simulate and verify concurrent
and distributed programs. It is a partial reimplementation of the \spn{}
model checker and supports a large subset of \prm{}, the modeling
language used by \spn{}. The \prg{} distribution contains a
comprehensive User's Guide and software documentation; this document is
intended to be a quick introduction to working with \prg{}.

\prg{} is run from the command line. \eui{} is a development environment
for \prg{} with a graphical user interface; it can be downloaded from
the \textsc{jSpin} project in Google Code
\url{http://code.google.com/p/jspin/}. A separate Quick Start Guide for
running \prg{} using \eui{} is available at that site.

The \prg{} software is copyrighted under the \textsc{GNU}
General Public License. The copyright statement and the text of the
license are included in the distribution.

Introductory textbooks on concurrency and model checking are:

\begin{itemize}
\item M. Ben-Ari. \textit{Principles of Concurrent and Distributed
Programming (Second Edition)}. Addison-Wesley, 2006.
\item M. Ben-Ari. \textit{Principles of the Spin Model Checker}.
Springer, 2008.
\end{itemize}

\section{Installation and execution}

This section describes installation and execution of \prg{} under the
Windows operating system. For other systems, see the User's Guide.

Download the archive \p{erigone-N.zip} (where \p{N} is the latest
version number) from the \p{Erigone} project at Google Code
\url{http://code.google.com/p/erigone/} and open the archive to a clean
directory. The executable file is \p{erigone.exe}, and there are
subdirectories containing the documentation and example programs. To run
\prg{}, execute the command:

\begin{verbatim}
erigone [arguments] filename
\end{verbatim}
where \p{filename} is the name of the file containing the \prm{} source
code; the default extension is \p{pml}. The command-line arguments can
be displayed by \p{erigone -h}.

\section{Simulating a \prm{} program}

This section show how to simulate two \prm{} programs for the mutual
exclusion problem; the first has an error that causes mutual exclusion
to be violated, whereas mutual exclusion always holds in the second
program.

\subsection*{The ``Second Attempt''}

The first program is the ``Second Attempt'' to solve the mutual
exclusion problem (Ben-Ari, 2006, Section~3.6) and is contained in
the file \p{second.pml} in the \p{examples} directory:
\begin{verbatim}
bool wantp = false, wantq = false;
byte critical = 0; 

active proctype p() {
    do 
    :: !wantq;
       wantp = true;
       critical++;
       assert (critical == 1);
       critical--;
       wantp = false;
    od
}

active proctype q() { /* Symmetrical */ }
\end{verbatim}

Run:
\begin{verbatim}
erigone examples\second
\end{verbatim}
The output will be similar to:
\begin{verbatim}
simulation terminated=assert statement is false,
  line=22,statement={assert (critical == 1)},
\end{verbatim}
because it is highly likely that a run will cause mutual exclusion to
be violated.

There are a large number of options for tracing the execution of \prg{}.
For example:
\begin{verbatim}
erigone -dm examples\second
\end{verbatim}
displays the states of the simulation:
\begin{verbatim}
initial state=0,p=1,q=1,wantp=0,wantq=0,critical=0,
next state=1,p=1,q=2,wantp=0,wantq=0,critical=0,
  ...
next state=16,p=3,q=3,wantp=1,wantq=1,critical=0,
next state=17,p=3,q=4,wantp=1,wantq=1,critical=1,
next state=18,p=4,q=4,wantp=1,wantq=1,critical=2,
simulation terminated=assert statement is false,
  line=22,statement={assert (critical == 1)},
\end{verbatim}

Since the variable \p{critical} has the value~2 when the assert
statement in line~22 of the program is evaluated in state~18, an error
has occurred and the simulation terminates.

All the output of \prg{} is in a uniform format that is easy to read and
easy to process by other tools. Each line consists of a sequence of
comma-terminated named associations of the form \p{"name=value,"}.

\subsection*{Mutual exclusion with a semaphore}

The second program uses a semaphore to achieve mutual exclusion
and is contained in the file \p{sem.pml} in the \p{examples} directory:
\begin{verbatim}
byte sem = 1;
byte critical = 0;

active proctype p() {	
  do :: 
    atomic {
      sem > 0;
      sem--
    }
    critical++;
    assert(critical == 1);
    critical--;
    sem++
  od
}

active proctype q() {	/* The same */ }

active proctype r() {	/* The same */ }	
\end{verbatim}

Run:
\begin{verbatim}
erigone examples\sem
\end{verbatim}
to simulate the program. The simulation will run until the step limit is
reached:
\begin{verbatim}
simulation terminated=too many steps,
  line=11,statement={assert(critical == 1)},
\end{verbatim}
You can run with the \p{-dm} argument to display the (very large number
of) states. Enter \bu{ctrl-c} to stop the simulation prematurely.

These simulations were \emph{random simulations}, where the
(nondeterministic) selection of the next statement to execute is chosen
randomly. See the User's Guide for information on how to run an
\emph{interactive simulation} where you control the section of
statements. \emph{Guided simulation} is explained in the next section.

\section{Verifying a \prm{} program with assert statements}

Let us verify these two programs. Run:
\begin{verbatim}
erigone -s examples\second
\end{verbatim}
where the argument \p{-s} means to run a verification in \emph{Safety mode}.
The output will be:
\begin{verbatim}
verification terminated=assert statement is false,
  line=22,statement={assert (critical == 1)},
\end{verbatim}
Now run:
\begin{verbatim}
erigone -g examples\second
\end{verbatim}
to run a \emph{guided simulation} using the trail which describes the
\emph{counterexample}, the sequence of states found during the
verification that leads to a violation of mutual exclusion. Add the
argument \p{-dm} to display the states in the counterexample.

For the semaphore program, running:
\begin{verbatim}
erigone -s examples\sem
\end{verbatim}
gives:
\begin{verbatim}
verification terminated=successfully,
\end{verbatim}

There are numerous options for tracing the verification; for example,
the argument \p{-dh} will print out which states were generated during
the search and stored in the hash table (or not stored because they were
already there):

\begin{verbatim}
inserted=true,p=1,q=1,wantp=0,wantq=0,critical=0,
inserted=true,p=2,q=1,wantp=0,wantq=0,critical=0,
inserted=true,p=3,q=1,wantp=1,wantq=0,critical=0,
inserted=true,p=4,q=1,wantp=1,wantq=0,critical=1,
inserted=true,p=5,q=1,wantp=1,wantq=0,critical=1,
inserted=true,p=6,q=1,wantp=1,wantq=0,critical=0,
inserted=false,p=1,q=1,wantp=0,wantq=0,critical=0,
  ...
\end{verbatim}

\section{Verifying a safety property using LTL}

\prg{} can verify a correctness property written as a formula in
\emph{linear temporal logic (LTL)}. There are two methods of specifying
an LTL property: internally within the program or externally in a
separate file. It is easier to use the first method; see the User's
Guide for information on using LTL properties in files.

The program in the file \p{second-ltl.pml} is the same as that in
\p{second.pml} except that the assert statements have been removed and
replace by an LTL formula for the correctness property:

\begin{verbatim}
ltl { [](critical<=1) }
\end{verbatim}
Read this as: the value of \p{critical} is \emph{always} (\p{[]}) less
than or equal to 1.

Run:
\begin{verbatim}
  erigone -s -t examples\second-ltl
\end{verbatim}
where the argument \p{-t} tells \prg{} to use the LTL formula. The
result is:
\begin{verbatim}
verification terminated=never claim terminated,
  line=9,statement={critical--},
\end{verbatim}

The phrase \emph{never claim terminated} is a technical term used by the
model checker to report that an error has been found. The counterexample
can be displayed by running a guided simulation.

Running a verification for the program with the semaphore gives the
result that the verification was terminated successfully.

\section{Verifying a liveness property using LTL}

Consider the following program, which is a very simple example used to
demonstrate the concept of fairness:

\begin{verbatim}
byte n = 0;
bool flag = false;

active proctype p() {
  do
  ::  flag -> break;
  ::  else -> n = 1 - n;
  od
}

active proctype q() {
  flag = true
}
\end{verbatim}

If process \p{q} ever executes, it will set \p{flag} to \p{true} and
process \p{p} will then terminate, but in an unfair computation (where
\p{q} never executes) this may never happen. Let us now try to prove the
property \p{<>flag} meaning that \emph{eventually} \p{flag} becomes
true. The \prm{} program can be found file \p{fair1.pml} and it
contains the LTL formula:

\begin{verbatim}
ltl { <>flag }
\end{verbatim}

Run:
\begin{verbatim}
erigone -a -t examples\fair1
\end{verbatim}

where \p{-a} means perform a verification in \emph{Acceptance mode}. In
this mode, the model checker searches for an \emph{acceptance cycle},
which is the technical term for an error in the verification of an LTL
formula containing \emph{eventually} (\p{<>}). The result is:

\begin{verbatim}
verification terminated=acceptance cycle,
  line=7,statement={else -> n = 1 - n},
\end{verbatim}

Run a guided simulation with the \p{-dm} argument:
\begin{verbatim}
initial state=0,p=1,q=1,n=0,flag=0,
start of acceptance cycle=1,
next state=1,p=3,q=1,n=0,flag=0,
next state=2,p=1,q=1,n=1,flag=0,
next state=3,p=3,q=1,n=1,flag=0,
next state=4,p=1,q=1,n=0,flag=0,
next state=5,p=3,q=1,n=0,flag=0,
simulation terminated=end of acceptance cycle,
line=7,statement={else -> n = 1 - n},
\end{verbatim}

You can see that state 5 is the same as state 1 so the computation can
continue indefinitely by cyclically repeating these five states, while
\p{flag} is false (= 0) in all the states. This happened because the
path is constructed by taking transitions only from process \p{p}.

Perform a verification in \emph{Acceptance mode with weak fairness}
by using the argument \p{-f}.

\begin{verbatim}
erigone -f -t examples\fair1
\end{verbatim}

In this mode, only weakly fair computations are candidates for
acceptance cycles. In a weakly fair computation, continually enabled
transitions must eventually be taken. Since the assignment statement
\p{flag=true} is obviously continually enabled, it must be taken. When
the verification is run, the result is: The result is:

\begin{verbatim}
verification terminated=successfully,
\end{verbatim}

\end{document}
