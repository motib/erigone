\documentclass[11pt]{article}
\usepackage{mathpazo}
\usepackage{url}
\usepackage{verbatim}

\newcommand*{\prg}{\textsc{Erigone}}
\newcommand*{\trc}{\textsc{Trace}}
\newcommand*{\lst}{\textsc{List}}
\newcommand*{\vmc}{\textsc{VMC}}
\newcommand*{\cmp}{\textsc{Compiler}}
\newcommand*{\spn}{\textsc{Spin}}
\newcommand*{\prm}{\textsc{Promela}}
\newcommand*{\ada}{\textsc{Ada 2005}}
\newcommand*{\gnat}{\textsc{gnat}}
\newcommand*{\dt}{\textsc{dot}}
\newcommand*{\smc}{\textit{SMC}}
\newcommand*{\p}[1]{\texttt{#1}}

\textwidth=15cm
\textheight=22.5cm
\topmargin=0pt
\headheight=0pt
\oddsidemargin=1cm
\headsep=0pt
\renewcommand{\baselinestretch}{1.1}
\setlength{\parskip}{0.20\baselineskip plus 1pt minus 1pt}
\parindent=0pt

\title{The \prg{} Model Checker\\\bigskip%
User's Guide\\\bigskip%
\large Version 3.2.5\\\bigskip}     %%%%% Change the version number!!!
\author{Mordechai (Moti) Ben-Ari\\
Department of Science Teaching\\
Weizmann Institute of Science\\
Rehovot 76100 Israel\\
\textsf{http://stwww.weizmann.ac.il/g-cs/benari/}}
%\date{}
\begin{document}
\maketitle
\thispagestyle{empty}

\vfill

\begin{center}
Copyright \copyright{} 2007-12 by Mordechai (Moti) Ben-Ari.
\end{center}
This work is licensed under the Creative Commons Attribution-ShareAlike 3.0
License. To view a copy of this license, visit
\url{http://creativecommons.org/licenses/by-sa/3.0/}; or, (b) send a letter
to Creative Commons, 543 Howard Street, 5th Floor, San Francisco,
California, 94105, USA.


\newpage

\section{Introduction}

\prg{} is a partial reimplementation of the \spn{} Model Checker. The goal
is to facilitate learning concurrency and model checking.
\begin{itemize}
\item \prg{} is single, self-contained, executable file so that
installation and use are trivial.
\item \prg{} produces a detailed trace of the model checking
algorithms. The contents of the trace are customizable and a uniform
keyword-based format is used that can be directly read or used by other
tools.
\item Extensive modularization is used in the design of the \prg{}
program to facilitate understanding the source code. This will also
enable researchers to easily modify and extend the program.
\end{itemize}
\prg{} implements a large subset of \prm{} that is sufficient for
demonstrating the basic concepts of model checking for the verification
of concurrent programs. No language constructs are added so that
programs for \prg{} can be used with \spn{} when more expressiveness and
better performance are desired.

\prg{} is written in \ada{} for reliability, maintainability and
portability. No non-standard constructs are used.

The \prg{} software is copyrighted under the \textsc{GNU} General Public
License. The copyright statement and the text of the license are
included in the distribution archive.

\subsection*{Acknowledgements}
I would like to thank Gerard J. Holzmann for his generous assistance
throughout the development of this project. The original compiler was
developed by Trishank Karthik Kuppusamy under the supervision of Edmond
Schonberg.

\section{References}
\begin{itemize}
\item Christel Baier and Joost-Pieter Katoen.
\textit{Principles of Model Checking}. MIT Press, 2008.
\item M. Ben-Ari. \textit{Ada for Software Engineers
(Second Edition with Ada 2005)}. Springer, 2009.
\item M. Ben-Ari. \textit{Principles of the Spin Model Checker}.
Springer, 2008.\\
The sample programs from the book have been adapted for \prg{}
and can downloaded as \p{psmc-erigone.zip}.
\item M. Ben-Ari. \textit{Principles of Concurrent and Distributed Programming (Second
Edition)}. Addison-Wesley, 2006.
\item Gerard J. Holzmann. \textit{The Spin Model Checker: Primer
and Reference Manual}.\\Addison-Wesley, 2004. The abbreviation
\smc{} is used to refer to this book.
\item Fred Kr\"{o}ger and Stephan Merz.
\textit{Temporal Logic and State Systems}. Springer, 2008.
\end{itemize}

\smallskip

\begin{center}
\begin{tabular}{l@{\hspace{3em}}l}
\hline
\prg{} & \url{http://stwww.weizmann.ac.il/g-cs/benari/erigone/}\\
& \url{http://code.google.com/p/erigone/}\\
\gnat{} & \url{https://libre.adacore.com/}\\
\spn{} & \url{http://spinroot.com/}\\
\hline
\end{tabular}
\end{center}

\section{Installation, building and execution}
\begin{itemize}

\item \textbf{Installation}\\Download the archive \p{erigone-N.zip} from
Google Code. Open the archive to a clean directory. If you wish to
examine or modify the source code, download the the archive
\p{erigone-source-N.zip} and open it to the same directory. The
following subdirectories are created:

\begin{itemize}
\item \p{docs}: documentation;
\item \p{examples}: \prm{} source code of example programs;
\item \p{src}: the source code of \prg{};
\item \p{trace}: the source code of the \trc{} program;
\item \p{list}: the source code of the \lst{} program;
\item \p{vmc}: the source code of the \vmc{} program;
\item \p{vmc-examples}: examples for the \vmc{} program.
\end{itemize}

The file \p{readme.txt} in \p{examples} describes the commands and
expected output for simulating and verifying the \prm{} programs in the
directory.

\item \textbf{Execution}
\begin{verbatim}
erigone [arguments] filename
\end{verbatim}

In addition to the \prg{} model checker, the archive contains additional
programs described in separate sections:

\begin{itemize}
\item \cmp{}: Run the compiler by itself (Section~\ref{s.compile});
\item \lst{}: Format the output of the compiler (Section~\ref{s.lst});
\item \trc{}: Format the output of the model checker (Section~\ref{s.trace});
\item \vmc{}: Generate graphs of the model checking algorithm
  (Section~\ref{s.vmc}).
\end{itemize}

\item \textbf{File names}\\
If an extension is not given in \p{filename}, the default \prm{} source
file extension is \p{pml}. Other file names are obtained by using the
file root and predefined extensions. The output of a compilation is the
\emph{automata file} with extension \p{aut} (AUTomata) that contains the
symbols and transitions resulting from the compilation. The extension of
the trail file that contains a counterexample for an unsuccessful
verification is \p{trl} (TRaiL). If the LTL correctness property
\p{filename} is not given, the default is the root with extension
\p{prp} (correctness PRoPerty).

\item \textbf{Building}\\
\prg{} was built with the \gnat{} \ada{} compiler. The source code is
divided into several subdirectories and the \gnat{} Project Manager is
used to build the program. The file \p{erigone.gpr} describes the
project configuration. Object files are placed in a subdirectory of
\p{src} called \p{obj} that must exist before executing the project
manager.

To build, execute:
\begin{verbatim}
gnatmake -Perigone [-O1] [-gnatn] [-g]
\end{verbatim}

The \p{-g} argument is necessary if you want to use the \p{gdb}
debugger. For a production build, optimization level \p{-O1}
significantly improves the performance, while \p{-gnatn} enables
inlining of subprograms.

The program \cmp{} is built within the same directory using the project
configuration file \p{compiler.gpr}. The program \lst{}, \trc{} and
\vmc{} are built within their own directories using the appropriate
\p{gpr} files.

\end{itemize}

\section{Command-line arguments}

\textbf{Execution mode}
\begin{center}
\begin{tabular}{|l|p{.45\textwidth}|}
\hline
\p{-r} & [R]andom simulation (default)\\\hline
\p{-i} & [I]nteractive simulation\\\hline
\p{-g} & [G]uided simulation\\\hline
\p{-gN} & [G]uided simulation using the N'th trail\\\hline\hline
\p{-s} & Verification of [s]afety\\\hline
\p{-a} & Verification of [a]cceptance\\\hline
\p{-f} & Verification with [f]airness\\\hline\hline
\p{-c} & [C]ompilation only\\\hline
\end{tabular}
\end{center}

\pagebreak[3]

\textbf{Execution limits (with defaults)}
\begin{center}
\begin{tabular}{|l|p{.4\textwidth}|r|}
\hline
\p{-lhN} & [H]ash slots & 22\\\hline
\p{-llN} & [L]ocation stack & 3\\\hline
\p{-lpN} & [P]rogress steps & 1\\\hline
\p{-lsN} & [S]tate stack & 2\\\hline
\p{-ltN} & [T]otal steps & 10\\\hline
\end{tabular}
\end{center}

\medskip

The parameter \p{N} is in thousands (except for hash slots). For the
stacks, the value is the number of \textbf{stack frames} and not the
number of \emph{bytes}. \emph{Total steps} is the number of steps of a
simulation or verification before the execution is terminated. For a
verification, a progress report can be printed after every
\emph{progress steps} have been executed if \p{g} is included in the
display arguments. The hash table is allocated $2^{N}$ slots, where $16
\leq N \leq 32$.

\medskip

\textbf{Additional execution arguments}
\begin{center}
\begin{tabular}{|l|p{.45\textwidth}|}
\hline
\p{-h} & Display the [h]elp screen\\\hline
\p{-mN} & Stop after the [m]'th error\\\hline
\p{-nN} & Ra[n]dom seed\\\hline
\p{-t[-][f]} & Read L[T]L formula\\\hline
\end{tabular}
\end{center}

If \p{-m0} is used, \emph{all} errors are reported and numbered trail
files are written for each one. Random numbers are used for random
simulation and for search diversity (Section~\ref{s.search}). The
default seed for random simulation is obtained from the clock.

The \p{-t} parameter is used to specify that an LTL formula is to be
used in a verification. The formula may be embedded in the \prm{} source
code or read from a file:
\begin{verbatim}
-t            use "ltl { ... }" if it exists
-t            otherwise, use default file "filename.prp"
-t-name       use "ltl name { ... }"
-tname.prp    use named file "name.prp"
\end{verbatim}

\medskip

\textbf{Display arguments}

The argument \p{-d} displays all the data. Selective display is possible
with the argument \p{-dX}, where \p{X} is a string of one or more
character taken from the following tables.

Compile and runtime messages:
\begin{center}
\begin{tabular}{|l|p{.45\textwidth}|}
\hline
\p{g} & Pro[g]ress messages\\\hline
\p{p} & [P]rogram (transitions and symbols)\\\hline
\p{r} & [R]un-time statistics\\\hline
\p{v} & [V]ersion and copyright notice\\\hline
\end{tabular}
\end{center}

\pagebreak[3]

Transitions:
\begin{center}
\begin{tabular}{|l|p{.45\textwidth}|}
\hline
\p{a} & [A]ll transitions from a state\\\hline
\p{c} & [C]hosen transition (simulation)\\\hline
\p{e} & [E]xecutable transitions from a state\\\hline
\p{l} & [L]ocation stack\\\hline
\p{t} & [T]rail\\\hline
\p{y} & B[y]te code\\\hline
\end{tabular}
\end{center}

\pagebreak[3]

States:
\begin{center}
\begin{tabular}{|l|p{.45\textwidth}|}
\hline
\p{h} & State in the [h]ash table\\\hline
\p{m} & States in a si[m]ulation\\\hline
\p{o} & Channel c[o]ntents\\\hline
\p{s} & [S]tate stack\\\hline
\end{tabular}
\end{center}

LTL translation:
\begin{center}
\begin{tabular}{|l|p{.45\textwidth}|}
\hline
\p{b} & [B]\"{u}chi automaton\\\hline
\p{n} & [N]odes in the LTL tableau\\\hline
\end{tabular}
\end{center}

\medskip

\textbf{Debug arguments}

The argument \p{-u} displays debugging data that will be of interest
when modifying the program. Selective display is possible
with the argument \p{-uX}, where \p{X} is a string of one or more
character taken from the following tables.

\begin{center}
\begin{tabular}{|l|p{.45\textwidth}|}
\hline
\p{c} & [C]ompiler tokens \\\hline
\p{i} & [I]nterpreter stack \\\hline
\p{r} & P[r]reprocessor\\\hline
\end{tabular}
\end{center}


\section{The subset of \spn{} that is implemented}
\subsection{\prm{} subset}

All constructs of the \prm{} language are implemented except for:
\begin{itemize}
\item \p{unsigned} and \p{typedef}.
\item Macros except for \p{\#define}.\footnote{One line only; continuation lines are
not supported.}
\item Progress labels and never claims except those generated by LTL formulas.
\item The specifiers \p{hidden}, \p{local}, \p{xr}, \p{xs} are
ignored but don't cause compilation errors.
\item \p{priority}, \p{unless}.
\item C code insertion.
\end{itemize}

There are limitations on certain other constructs:
\begin{itemize}
\item Bitwise and shift operators are supported only for one-byte types. 
\item \p{d\_step} is implemented like \p{atomic}.
\item Only the first form of the \p{for} statement (iteration over
bounds given by expressions) is implemented.
\item You cannot initialize local variables declared after a
statement. (This is likely to happen if an \p{inline} declares a
local variable.)
\end{itemize}

Limits on the size of the \prm{} models are compiled into \prg{}
(Appendix~\ref{a.limits}). If you need to verify larger models, you will
need to rebuild the software.

\subsection{Search diversity}\label{s.search}

By default, the state space of a model is searched in a fixed order.
\prg{} supports \emph{search diversity} where randomness is introduced
into the search by specifying a seed \p{-nN} for a
verification.\footnote{For an introduction to search diversity and an
explanation of the examples, see: M. Ben-Ari and F. Kaloti-Hallak,
Demonstrating random and parallel algorithms with Spin, \textit{ACM
Inroads}, in press.} For example, consider the Tseitin clauses
associated with the graph $K_{4,4}$ given in the file \p{sat4-unsat} in
the directory \p{examples}; this set of clauses is unsatisfiable and a
safety verification will prove this in 1835006 steps. The file
\p{sat4-sat} contains the same set of clauses with the parity of one
literal changed so that the set is satisfiable. With the seed \p{-n91},
the counterexample (satisfying interpretation) is found in 127389 steps,
while with the seed \p{-n58} only 361 steps are needed.

The Java program \p{GenerateSatERI} can be used to generate Tseitin
clauses corresponding to $K_{n,n}$ together with random satisfiable
instances and batch files with random seeds. However, \prg{} is limited
to sets of clauses for $n\leq 4$.  

\newpage

\section{Display format}

Data is displayed in a uniform format: lines of named associations, each
of which is terminated with a comma. This format is verbose, but easy to
read and easy to parse by postprocessors. \footnote{In the following
description, the display option needed to obtain each item is given in
parentheses and lines have been elided and reformatted to fit the page.
Furthermore, the display of later versions might be somewhat different
from what appears here.}

\subsection{Data structure display}

The display begins with a title line (\p{v}) and a line
with the parameters of the execution (\p{r}):

\begin{footnotesize}
\begin{verbatim}
Erigone v3.1.0, Copyright 2008-11 by Moti Ben-Ari, GNU GPL.
execution mode=simulation,simulation mode=random,seed=-1,trail number=0,total steps=10,
\end{verbatim}
\end{footnotesize}

The symbol table (\p{p}) includes for each variable its type, size,
length (for arrays), flags (scope is global or local, parameter or not),
the offset into its frame in the state vector and the byte code (\p{y})
for initialization:

\begin{footnotesize}
\begin{verbatim}
variables=6,
name=n,type=byte_type,offset=0,length=1,size=1,scope=0,
  parameter=0,byte code={load_const 0 0,byte_store 0 0,},
name=finished,type=byte_type,offset=1,length=1,size=1,scope=0,
  parameter=0,byte code={load_const 0 0,byte_store 1 0,},
name=P.i,type=byte_type,offset=0,length=1,size=1,scope=1,
  parameter=0,byte code=,
name=P.temp,type=byte_type,offset=1,length=1,size=1,scope=1,
  parameter=0,byte code=,
name=Q.i,type=byte_type,offset=0,length=1,size=1,scope=1,
  parameter=0,byte code=,
name=Q.temp,type=byte_type,offset=1,length=1,size=1,scope=1,
  parameter=0,byte code=,
symbol table end=,
\end{verbatim}
\end{footnotesize}

Tables of numeric and string constants (if any) are also
displayed when \p{p} is given.

The frame table gives the offset of the frame for global and local
variables:
\begin{footnotesize}
\begin{verbatim}
frame table start=,
global=,offset=0,
pid=0,offset=2,
pid=1,offset=4,
pid=2,offset=6,
frame table end=,
\end{verbatim}
\end{footnotesize}

For each process, a table of its transitions is display containing: the
source and target state, various flags, the source code and the byte
code:

\begin{footnotesize}
\begin{verbatim}
transitions start=,
processes=3,
process=P,initial=1,transitions=7,
number=0,source=1,target=3,atomic=0,end=0,accept=0,line=7,
  statement={(i> 10)},byte code={byte_load 2 0,iconst 10 0,icmpgt 0 0,},
number=1,source=1,target=4,atomic=0,end=0,accept=0,line=8,
  statement={else},byte code={logic_else 0 0,},
number=2,source=3,target=9,atomic=0,end=0,accept=0,line=13,
  statement={finished++},byte code={iinc 1 1,},
number=3,source=4,target=5,atomic=0,end=0,accept=0,line=9,
  statement={temp=n},byte code={byte_load 0 0,byte_store 3 0,},
  ...
transitions end=,
\end{verbatim}
\end{footnotesize}

\subsection{LTL translation display}
The display of the translation of the LTL formula to a B\"{u}chi
automaton (BA) appears before that of the transitions of the processes,
because the transitions of the BA are added to the transition table as a
never claim. The display starts with the LTL formula together with the
formula obtained by pushing negation inward (\p{b}). This is followed by
the nodes of the tableau (\p{n}):

\begin{footnotesize}
\begin{verbatim}
ltl formula=![]<>csp,
push negation=<>[]!csp,
nodes start=,
expanding=,node=1,incoming={0,},new={<>[]!csp,},old=,next=,with set=,
expanding=,node=2,incoming={0,},new={[]!csp,},old={<>[]!csp,},next=,with set=,
expanding=,node=4,incoming={0,},new={!csp,},old={[]!csp,<>[]!csp,},next={[]!csp,},with set=,
expanding=,node=4,incoming={0,},new=,old={!csp,[]!csp,<>[]!csp,},next={[]!csp,},with set=,
expanding=,node=5,incoming={4,},new={[]!csp,},old=,next=,with set={4,},
expanding=,node=6,incoming={4,},new={!csp,},old={[]!csp,},next={[]!csp,},with set={4,},
expanding=,node=6,incoming={4,},new=,old={!csp,[]!csp,},next={[]!csp,},with set={4,},
  ...
exists=,node=6,new incoming={4,6,},
  ...
nodes end=,
\end{verbatim}
\end{footnotesize}

A BA is extracted from the tableau (\p{b}) and optimized:

\begin{footnotesize}
\begin{verbatim}
optimized buchi automaton start=,
source=0,target=0,atomic=0,end=0,accept=0,line=0,statement={1},
  byte code={iconst 1 0,},
source=0,target=4,atomic=0,end=0,accept=0,line=0,statement={!csp},
  byte code={bit_load 3 0,logic_not 0 0,},
source=4,target=4,atomic=0,end=0,accept=1,line=0,statement={!csp},
  byte code={bit_load 3 0,logic_not 0 0,},
optimized buchi automaton end=,
\end{verbatim}
\end{footnotesize}

The BA is used to create a never claim process in the transition table
(\p{p}). In addition, the set of accept states is identified to make
them easier to find.

\subsection{Display of a simulation}
A simulation begins from an initial state (\p{m}):

\begin{footnotesize}
\begin{verbatim}
initial state=,P=1,Q=1,Finish=1,n=0,finished=0,P.i=1,P.temp=0,Q.i=1,Q.temp=0,
\end{verbatim}
\end{footnotesize}

For each step of the simulation, the display can contain the set of all
transitions from the state (\p{a}), the set of executable transitions
(\p{e}), the chosen transition (\p{c}) and the next state in the
simulation (\p{m}):

\begin{footnotesize}
\begin{verbatim}
all transitions=5,
process=P,source=1,target=3,atomic=0,end=0,accept=0,line=7,
  statement={(i> 10)},byte code={byte_load 2 0,iconst 10 0,icmpgt 0 0,},
process=P,source=1,target=4,atomic=0,end=0,accept=0,line=8,
  statement={else}, byte code={logic_else 0 0,},
process=Q,source=1,target=3,atomic=0,end=0,accept=0,line=19,
  statement={(i> 10)},byte code={byte_load 4 0,iconst 10 0,icmpgt 0 0,},
process=Q,source=1,target=4,atomic=0,end=0,accept=0,line=20,
  statement={else},byte code={logic_else 0 0,},
process=Finish,source=1,target=2,atomic=0,end=0,accept=0,line=29,
  statement={finished== 2},byte code={byte_load 1 0,iconst 2 0,icmpeq 0 0,},

executable transitions=2,
process=P,source=1,target=4,atomic=0,end=0,accept=0,line=8,
  statement={else},byte code={logic_else 0 0,},
process=Q,source=1,target=4,atomic=0,end=0,accept=0,line=20,
  statement={else},byte code={logic_else 0 0,},

chosen transition=,
process=Q,source=1,target=4,atomic=0,end=0,accept=0,line=20,
  statement={else},byte code={logic_else 0 0,},

next state=,P=1,Q=4,Finish=1,n=0,finished=0,P.i=1,P.temp=0,Q.i=1,Q.temp=0,
\end{verbatim}
\end{footnotesize}

To display the data in buffered channels use \p{-u}.
The first field is the current number of messages in the channel;
this followed by the messages themselves in brackets:

\begin{footnotesize}
\begin{verbatim}
next state=,Pinit=3,ch1=1,a=0,b=0,c=0,channel1={2,[1,2,3,],[4,5,6,],},
\end{verbatim}
\end{footnotesize}

The simulation can terminated by an assertion violation, or a valid or
invalid end state, or by exceeding the total steps allowed:

\begin{footnotesize}
\begin{verbatim}
simulation terminated=valid end state,
\end{verbatim}
\end{footnotesize}

The runtime statistics are then displayed (\p{r}):

\begin{footnotesize}
\begin{verbatim}
steps=87,
times=,compilation=0.33,simulation=0.18,
\end{verbatim}
\end{footnotesize}

\subsection{Display of a verification}

During a verification, the sets of all transitions (\p{a}) and
executable transitions (\p{e}) can be displayed as for a simulation.
Operations on the stacks can be displayed, both for the state stack
(\p{s}):

\begin{footnotesize}
\begin{verbatim}
push state=10,p=6,q=6,wantp=1,wantq=1,critical=0,
  ...
top state=10,p=6,q=6,wantp=1,wantq=1,critical=0,
  ...
pop state=10,reason=no_more_transitions,
\end{verbatim}
\end{footnotesize}

and for the location stack (\p{l}):

\begin{footnotesize}
\begin{verbatim}
push transition=16,process=0,transition=5,never=0,visited=false,last=false,
  ...
top transition=16,process=0,transition=5,never=0,visited=true,last=false,
  ...
pop transition=16,
\end{verbatim}
\end{footnotesize}

An attempt is made to insert each new state into the hash table (\p{h});
it may succeed or fail:

\begin{footnotesize}
\begin{verbatim}
inserted=true,p=5,q=3,wantp=1,wantq=1,critical=1,
  ...
inserted=false,p=6,q=3,wantp=1,wantq=1,critical=0,
\end{verbatim}
\end{footnotesize}

For verification with acceptance, the ``inner'' flag will be displayed
indicating if the state is part of the outer or inner search; the
``seed'' state is also displayed:

\begin{footnotesize}
\begin{verbatim}
seed=,p=4,q=13,:never:=4,wantp=1,wantq=1,turn=2,csp=0,inner=1,fair=3,
  ...
inserted=true,p=14,q=11,:never:=4,wantp=1,wantq=1,turn=2,csp=1,inner=1,fair=1,
\end{verbatim}
\end{footnotesize}

For verification with fairness, the counter of the copies of the states
is displayed

The outcome and the runtime statistics (\p{r}) are
displayed at the end of a verification:

\begin{footnotesize}
\begin{verbatim}
verification terminated=never claim terminated,

steps=47,
state stack elements=9,element size=22,memory=198,
transition stack elements=14,element size=5,memory=70,
states stored=19,matched=9,total=28,element size=22,memory=418,
times=,compilation=0.03,verification=0.09,
\end{verbatim}
\end{footnotesize}

Progress messages can be displayed during a long verification (\p{g}).

\newpage

\section{The \cmp{} program}\label{s.compile}

\cmp{} is a main program that runs the compiler as a separate program
producing the automata file:
\begin{verbatim}
  compiler promela-filename [automata-filename]
\end{verbatim}


\section{The \lst{} program}\label{s.lst}

The \lst{} program reads the automata file and formats it for display.
Run it after running the compiler:
\begin{verbatim}
compiler filename
list filename
\end{verbatim}
or
\begin{verbatim}
erigone -c filename
list filename
\end{verbatim}


\section{The \trc{} program}\label{s.trace}

The \trc{} program performs string processing on the output data written
by \prg{}. It prints the output of a simulation in tabular form. First
execute:
\begin{verbatim}
erigone -dcmop filename > filename.trc
\end{verbatim}
to write a file with the symbol table, states and transitions
taken. Then execute:
\begin{verbatim}
trace [arguments] filename
\end{verbatim}
The arguments are:
\begin{center}
\begin{tabular}{|l|l|}
\hline
\p{-tn} & Number of lines between column [t]itles\\\hline\hline
\p{-ln} & Column width for [l]ine numbers\\\hline
\p{-pn} & Column width for [p]rocesses\\\hline
\p{-sn} & Column width for [s]tatements\\\hline
\p{-vn} & Column width for [v]ariables\\\hline\hline
\p{-xs} & E[x]clude variables with \p{s}\\\hline
\p{-ms} & Exclude state[m]ents with \p{s}\\\hline
\end{tabular}
\end{center}
Here is an example of the output of this program:
\begin{footnotesize}
\begin{verbatim}
Trace v1.00, Copyright 2008-9 by Moti Ben-Ari, GNU GPL.
Proc Line Statement        wantp  wantq  critic 
                           0      0      0      
p    7    {!wantq}         0      0      0      
q    19   {!wantp}         0      0      0      
q    20   {wantq=true}     0      1      0      
p    8    {wantp=true}     1      1      0      
q    21   {critical++}     1      1      1      
p    9    {critical++}     1      1      2      
p    10   {assert(critical 1      1      2      
simulation terminated=assert statement is false,
\end{verbatim}
\end{footnotesize}

The \p{-x} and \p{-m} arguments are used for filtering the output.
The strings \p{s} are lists of strings terminated by \p{\#}:
\begin{verbatim}
-xwant# -massert#!#
\end{verbatim}

The meaning is that a column for a \emph{variable} (\p{-x}) or a row for
a \emph{statement} (\p{-m}) will not be printed if one of those strings
appears within the variable or statement, respectively. With the above
arguments, the columns for \p{wantp} and \p{wantq} will not be printed,
nor will the rows with the \p{assert} statements or the negation
operator.


\section{The \vmc{} program}\label{s.vmc}

\vmc{} (\textit{Visualization of Model Checking}) is a postprocessor of
the trace output of \prg{} that generates a sequence of graphs showing
the incremental building and traversing of the state space during a
simulation or verification. First, run \prg{} using
the display options shown in the examples below and
redirect the output to a file with extension \p{trc}; for the Third
Attempt, the commands for a simulation and a safety verification are:
\begin{verbatim}
erigone    -dehlmprs third > third.trc
erigone -s -dehlmprs third > third.trc
\end{verbatim}
Now run \vmc{}:\footnote{\vmc{} knows whether a simulation or
verification was run from the information in the \p{trc} file.} 
\begin{verbatim}
vmc third
\end{verbatim}
This will create a sequence of files in the \dt{} format of
\textsc{GraphViz}. The file names will have a sequence number appended.
Now, run the \dt{} program on each of these files. For Windows, the
following batch command will do this for each file, where the file name
is substituted for the parameter \verb+%1+ and we assume that \dt{} is
in the path:
\begin{verbatim}
for %%F in (%1-*.dot) do dot -Tpng %%F > %%~nF.png
\end{verbatim}
You can now display the graphs using software such as \p{IrfanView}.

A Windows batch file for the entire procedure is included in the archive.

The ``prologue'' of the \dt{} file (specifying the size of the nodes,
etc.) can be different for each program and is contained in the file
\p{filename.prg}. If this file does not exist, it is created with
default values.

\medskip

\textbf{Reading the graph}

Each node is rectangular and is labeled with:
\begin{itemize}
\item For each process, the line number and statement at the
location counter in this state. For a simulation, transitions that are
executable are decorated with the at-sign \verb+@+.
\item For each variable, its value. The order of the variables is the
same as that in which they are declared. (The order can be checked in
the \p{aut} file if there is any doubt.)
\item For a verification, an \emph{end state} is decorated with a sharp
\verb+#+ and an accept state is decorated with an asterisk \verb+*+.
\item For a fairness verification, the count of the copy of the state
is displayed within square brackets.
\end{itemize}

Visual effects are applied to states as follows:
\begin{itemize}
\item The node for the current state has an elliptical border.
\item A node for a state that is in error is colored red.
\item For a verification:
\begin{itemize}
\item Nodes for states on the stack have a bold border.
\item Nodes for matched states in the hash table have a double border.
\item Nodes for states in an inner search are filled in.
\end{itemize}
\item For a simulation:
\begin{itemize}
\item The number of borders of a node indicates the number of times
that the node has been visited.
\end{itemize}
\end{itemize}

The archive \p{vmc-examples.zip} contains \prm{} for three programs,
together with the \dt{} and \textsc{png} files that result from running
\vmc{} and \dt{}. The program \p{second} causes an assert statement to
evaluate to false, \p{third} has an invalid end state, and \p{fair}
contains an accept cycle if run with \p{-a} rather than \p{-f}.

\newpage

\appendix
\section{Size limits}\label{a.limits}

Here is a list of the limits compiled into \prg{} and the package
specifications where they appear. The declaration of the size of the
compressed state vector is in a separate package \p{Config\_State} to
enable a future implementation of per-model verifiers with minimal
recompilation. To restrict the coupling between the compiler and the
model checker, the limits marked \p{*} are declared independently in
\p{Compiler\_Global}.

\begin{center}
\begin{tabular}{|l|l|l|}\hline
\multicolumn{3}{|c|}{Declarations in package \p{Global}}\\\hline
Identifier & Type & Meaning\\\hline\hline
\p{Byte} & \p{mod 256} & Values and indices\\\hline
\p{Name*} & \p{String(1..32)} & Identifiers\\\hline
\p{Line*} & \p{String(1..128)} & Source statements, strings\\\hline
\multicolumn{3}{c}{}\\\hline
\multicolumn{3}{|c|}{Declarations in package \p{Config}}\\\hline
Identifier & Type & Meaning\\\hline\hline
\p{Process\_Index} & \p{Byte range 0..7} & Processes\\\hline
\p{Symbol\_Index} & \p{Byte range 0..31} & Symbols\\\hline
\p{Message\_Index} & \p{Byte range 0..3} & Elements in a channel message\\\hline
\p{Channel\_Index} & \p{Byte range 1..8} & Channels\\\hline
\p{Data\_Index} & \p{Byte range 0..63} & Bytes of data in a state\\\hline
\p{Transition\_Index*} & \p{Byte range 0..254} & Transitions in a process\\\hline
\p{Location\_Index} & \p{Byte range 0..15} & Transitions from a single state\\\hline
\p{Byte\_Code\_Index*} & \p{Byte range 0..255} & Byte codes per statement\\\hline
\p{Interpret\_Index} & \p{Byte range 0..63} & Interpretation stack\\\hline
\p{Node\_Index} & \p{Integer range 0..511} & Nodes in the state space diagram\\\hline
\p{Max\_Futures} & \p{constant := 4} & Number of future formulas\\\hline
\multicolumn{3}{c}{}\\\hline
\multicolumn{3}{|c|}{Declarations in package \p{Config\_State}}\\\hline
Identifier & Type & Meaning\\\hline\hline
\p{Process\_Size\_Index} & \p{Byte range 0..7} & Size of processes in a state vector\\\hline
\p{Variable\_Size\_Index} & \p{Byte range 0..31} & Size of variables in a state vector\\\hline
\multicolumn{3}{c}{}\\\hline
\multicolumn{3}{|c|}{Declarations in package \p{Compiler\_Global}}\\\hline
Identifier & Type & Meaning\\\hline\hline
\p{Table\_Index} & \p{Byte range 0..63} & Size of the symbol table\\\hline
\p{Token\_Index} & \p{Byte range 0..4095} & Number of tokens in the source\\\hline
\end{tabular}
\end{center}

\end{document}
